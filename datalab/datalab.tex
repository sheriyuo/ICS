\documentclass{noithesis}

\newtheorem{theorem}{定理}[section]
\newtheorem{definition}[theorem]{定义}
\newtheorem{lemma}[theorem]{引理}
\newtheorem{property}{性质}[theorem]

\begin{document}
	
\title{ICS~I~datalab~报告}
	\author{中国人民大学~~sheriyuo}
	
	\maketitle
	
	\begin{abstract}
		RUC 2023 计算机理论基础 I datalab 的详细实现和一定优化方案。
		
		不理解为什么要卷运算符个数,更不理解为什么要把去年最优解设为今年的 $95$ 分。
		
		独立完成一些 $95$ 分已耗费笔者太多精力,水平有限,告辞。排名 12/195。
\begin{verbatim}
bitXor 0.95 7/7/7/7
thirdBits 0.95 4/4/4/4
fitsShort 0.95 4/4/4/4
isTmax 0.95 5/5/5/6
fitsBits 1.90 5/5/6/7
upperBits 1.00 4/5/6/9
anyOddBit 1.90 7/7/7/7
byteSwap 1.90 10/10/10/17
absVal 3.80 3/3/3/5
divpwr2 1.90 5/5/6/7
float_neg 1.90 4/4/4/6
logicalNeg 3.80 5/5/5/6
bitMask 2.85 5/5/6/8
isGreater 2.85 7/7/9/12
logicalShift 2.85 5/5/6/14
satMul2 2.85 6/6/7/10
subOK 2.85 7/7/8/12
trueThreeFourths 3.80 9/9/11/12
isPower2 3.80 6/6/7/11
float_i2f 4.00 14/15/21/32
howManyBits 3.80 22/22/28/37
float_half 3.80 12/12/16/22
---
sum = 95.4310
\end{verbatim}
	\end{abstract}
	
	\section{bitXor}
	
	用 \verb|~| 和 \verb|&| 实现位运算异或。
	
	提取 $0$ 的贡献,运用德摩根律有
	
	\begin{displaymath}
		x\oplus y = \overline{xy \cup \overline{xy}} = \overline x\overline y \cap \overline{(\overline x\overline y)}
	\end{displaymath}

	需要 $7$ 个运算符,提取 $1$ 的贡献则需要 $8$ 个运算符。
	
\begin{lstlisting}
int bitXor(int x, int y) {
	return ~(x & y) & ~(~x & ~y);
}
\end{lstlisting}

	\section{thirdBits}
	
	求出从 LSB 开始,每个第 $3$ 位为 $1$ 其余位为 $0$ 的数。
	
	数字上限是 $0\sim 255$,所以采用初始值 \verb|0x49|,左移倍增两次即可,需要 $4$ 个运算符。
	
\begin{lstlisting}
int thirdBits(void) {
	int x = 0x49;
	x = x | (x << 9);
	x = x | (x << 18);
	return x;
}
\end{lstlisting}
	
	\section{fitsShort}
	\label{fitsShort}
	
	求出 $x$ 是否在 \verb|short| 范围内。
	
	若一个 \verb|int| 的值在 \verb|short| 范围内,那么符号位后 $16$ 位应该与符号位相同,再后 $15$ 位为 \verb|short| 有效位。
	
	运用算术右移,异或判断是否相等即可,需要 $4$ 个运算符。
	
\begin{lstlisting}
int fitsShort(int x) {
	x = (x >> 15) ^ (x >> 31);
	return !x;
}
\end{lstlisting}

	\section{isTmax}
	
	求出 $x$ 是否是 \verb|INT_MAX|,不能使用移位。
	
	若 \verb|x = INT_MAX|,那么 \verb|y = x + 1|,此时 \verb|y + y = 0|。
	
	那么判断 \verb|!(y + y)|,并且特判同时成立的 \verb|x = UINT_MAX| 即可,需要 $5$ 个运算符。
	
\begin{lstlisting}
int isTmax(int x) {
	int y = x + 1;
	return !(y + y + !y);
}
\end{lstlisting} 

	\section{fitsBits}
	\label{fitsBits}
	
	求出 $x$ 是否能用 $n(1\leq n\leq 32)$ 位二进制位存储。
	
	与 \ref{fitsShort} 相同,符号位后 $32 - n$ 位应该与符号位相同,再后 $n$ 位为有效位。
	
	运用算术右移,异或判断是否相等即可。
	
	注意到 \verb|x >> n - 1| 实际上需要 \verb|x >> n + ~0|,可以利用移位位数会 $\bmod 32$ 的性质,写成 \verb|x >> n + 31|,只需要 $5$ 个运算符。
	
\begin{lstlisting}
int fitsBits(int x, int n) {
	x = (x >> n + 31) ^ (x >> 31);
	return !x;
}
\end{lstlisting} 

	\section{upperBits}
	
	求出前 $n(0\leq n\leq 32)$ 高位二进制位为 $1$ 其余位为 $0$ 的数。
	
	先求出 \verb|1 << 31|,然后运用算数右移的性质右移 $n-1$ 位即可。
	
	同 \ref{fitsBits} 的优化,可以用 \verb|x >> n + 31| 省去一次取反。
	
	特判 $n=0$ 的情况,将 \verb|1 << 31| 替换为 \verb|!!n << 31| 即可,需要 $5$ 个运算符。
	
	\verb|!!n| 还存在优化空间,发现 \verb|(n + 31) & 32| 只会在 $n=0$ 时取 $0$,于是将 \verb|!!n << 31| 替换为 \verb|(n + 31 & 32) << 26| 即可,只需要 $4$ 个运算符。
	
\begin{lstlisting}
int upperBits(int n) {
	int x = n + 31;
	return (x & 32) << 26 >> x;
}
\end{lstlisting} 

	\section{anyOddBit}
	
	判断 $x$ 奇数位上是否有 $1$。
	
	用 \verb|0xaa| 左移倍增两次得到所有奇数位都为 $1$ 的数,取位运算与后转 \verb|bool| 即可。
	
\begin{lstlisting}
int anyOddBit(int x) {
	int y = 0xaa;
	y = y | (y << 8);
	y = y | (y << 16);
	return !!(y & x);
}
\end{lstlisting} 
	
	\section{byteSwap}
	
	求出 $x$ 交换第 $n,m(0\leq n,m\leq 3)$ 个字节后的数。
	
	利用异或结合律实现 \verb|swap|,将要交换的两个字节异或后放在最低有效字节上,取该字节左移分别 $n,m$ 位与原数作异或操作,即可得到交换之后的数,需要 $10$ 个运算符。
	
\begin{lstlisting}
int byteSwap(int x, int n, int m) {
	int y = 0xff;
	n = n << 3;
	m = m << 3;
	y = y & ((x >> n) ^ (x >> m));
	return x ^ ((y << n) | (y << m));
}
\end{lstlisting}

	\section{absVal}
	\label{absVal}
	
	求出 $x$ 的绝对值。
	
	若 $x$ 是负数,其绝对值即为 \verb|-x = ~(x - 1)|。而注意到 \verb|~| 和 \verb|-1| 采用同样的补码进行操作,提出 \verb|y = x >> 31| 来作为运算的同时也不影响正数的结果,答案为 \verb|(x + y) ^ y|,需要 $3$ 个运算符。
	
\begin{lstlisting}
int absVal(int x) {
	int y = x >> 31;
	return (x + y) ^ y;
}
\end{lstlisting}
	
	\section{divpwr2}
	
	求出 $x$ 除以 $2^n(0\leq n\leq 30)$ 向下取整的结果。
	
	正数的右移是向下取整的,而负数的右移是向上取整的,给负数加上 $2^n-1$ 的偏移量后右移 $n$ 位即可。
	
	一种偏移量的求法是,获取符号位 \verb|y = x >> 31|,用 \ref{absVal} 中 \verb|y| 也可以用来代替 \verb|-1| 的技巧,有 \verb|k = ((1 << n) + y) & y|,需要 $6$ 个运算符。
	
	使用 \verb|1 << n| 是可以优化的,利用 \verb|x >> 31| 的补码性质,有 \verb|k = y ^ (y << n)|,只需要 $5$ 个运算符。
	
\begin{lstlisting}
int divpwr2(int x, int n) {
	int y = x >> 31;
	int k = y ^ (y << n);
	return (x + k) >> n;
}
\end{lstlisting}

	\section{float\_neg}
	
	给定浮点数 $f$ 的 \verb|unsigned| 二进制表示,求出 $-f$ 的二进制表示。若 \verb|f = nan|,返回 \verb|nan|,可以使用所有运算符、\verb|unsigned| 及其范围内常数和 \verb|if|,\verb|while|。
	
	判断 \verb|nan| 只需要判断其数位是否有 $1$,即 \verb|nf & 0x7f800000 > 0x7f800000|。如果不是 \verb|nan|,异或 \verb|0x80000000| 修改符号位即可,需要 $4$ 个运算符。
	
\begin{lstlisting}
unsigned float_neg(unsigned uf) {
	if((uf & 0x7fffffff) > 0x7f800000)
	return uf;
	return uf ^ 0x80000000;
}
\end{lstlisting}
	
	\section{logicalNeg}
	
	使用去除 \verb|!| 的位运算符实现 \verb|!x|。
	
	考虑只有 $0$ 的负数为 $0$ 本身,所以 \verb|!x| 为假有 \verb!x | -x! 符号位为 $1$,提出符号位设为 $0$ 即可。
	
	可以直接 \verb!((x | ~x + 1) >> 31) + 1!,需要 $4$ 个运算符。
\begin{lstlisting}
int logicalNeg(int x) {
	return ((x | (~x + 1)) >> 31) + 1;
}
\end{lstlisting}

	\section{bitMask}
	
	求出二进制位第 $lowbit \sim highbit$ 位为 $1$ 的数,$lowbit> highbit$ 时为 $0$。
	
	可以通过 \verb|(~0 << lowbit) ^ (~0 << highbit << 1)| 的方法来得到这个数,但是该方法无法特判 $0$ 的情况,需要再与上一个 \verb|~0 << lowbit|,需要 $6$ 个运算符。
	
	考虑优化的本质是去除特判,所以不应该采用异或而是位运算与。对于 $highbit$,采用 \verb|~0 + (1 << highbit << 1)| 即可得到第 $0\sim highbit$ 位为 $1$ 的数,直接取与,此时仍需要 $6$ 个运算符。
	
	发现可以用 \verb|2 << highbit| 替换 \verb|1 << highbit << 1|,也回避了 \verb|<< 32| 的 ub,只需要 $5$ 个运算符。
\begin{lstlisting}
int bitMask(int highbit, int lowbit) {
	int b = ~0;
	return (b << lowbit) & (b + (2 << highbit));
}
\end{lstlisting}
	
	\section{isGreater}
	
	判断是否有 $x>y$。
	
	如果 $x,y$ 符号位相同,\verb|x + ~y| 的符号位为 $1$ 时有 $x\leq y$,符号位为 $0$ 时有 $x>y$,直接用 \verb|(x + ~y >> 31) + 1| 提取即可。
	
	如果 $x,y$ 符号位不同,直接判断是否有 $x> -1$ 即可,改为 \verb!x + ~(x ^ y >> 31 | y)!,需要 $7$ 个运算符。
\begin{lstlisting}
int isGreater(int x, int y) {
	return ((x + ~(((x ^ y) >> 31) | y)) >> 31) + 1;
}
\end{lstlisting}
	
	\section{logicalShift}
	
	对 \verb|int| 实现逻辑右移。
	
	算术右移 \verb|x >> n| 后,原最高位在第 $31-n$ 位,此时应将原最高位前的数位全部补 $0$。
	
	于是让 \verb|x >> n| 加上 \verb|1 << (31 - n)| 后再异或上 \verb|1 << (31 - n)| 即可。
	
	可以用 \verb|31 ^ n| 来实现 \verb|31 - n|,只需要 $5$ 个运算符。
\begin{lstlisting}
int logicalShift(int x, int n) {
	int t = 1 << (31 ^ n);
	return ((x >> n) + t) ^ t;
}
\end{lstlisting}
	
	\section{satMul2}
	
	求出 $x\times 2$,如果 $x\times 2$ 溢出超过 $Tmin$ 或 $Tmax$,将其赋值为 $Tmin$ 或 $Tmax$。
	
	溢出的条件是符号位不同,即 \verb|x ^ (x << 1) >> 31 = -1 = y|,否则 \verb|y = 0|。
	
	如果出现溢出,可以发现 \verb|y << y = -1 << -1 = 1 << 31| 本质上是在求 $Tmin$,此时 \verb|(x << 1) >> y = (x << 1) >> 31| 就是 $x\times 2$ 的符号位,如果溢出为负数为 $-1$,溢出为正数为 $0$。
	
	巧妙的是,$Tmin - 1 = Tmax$,于是答案即为 \verb|(y << y) + ((x << 1) >> y)|,只需要 $6$ 个运算符。
\begin{lstlisting}
int satMul2(int x) {
	int x2 = x << 1;
	int y = (x ^ x2) >> 31;
	return (y << y) + (x2 >> y);
}
\end{lstlisting}	
	
	\section{subOK}
	
	判断 $x-y$ 是否出现整型溢出。
	
	$x-y$ 出现整型溢出当且仅当 $x,y$ 符号位不同,且 $x,x-y$ 符号位也不同,直接用 \verb|(x ^ y) & (x ^ (x + ~y + 1))| 判断即可,需要 $8$ 个运算符。
	
	考虑优化,发现 $x-y$ 是否溢出跟 $y-x-1$ 是否溢出等价,可以优化掉取反后的 \verb|+ 1|,只需要 $7$ 个运算符。
\begin{lstlisting}
int subOK(int x, int y) {
	int z = y + ~x;
	return !(((x ^ y) & (y ^ z)) >> 31);
}
\end{lstlisting}

	\section{trueThreeFourths}
	
	计算 $x\times \frac 34$ 向 $0$ 取整后的结果。
	
	\verb|x + (~x >> 2)| 计算的是 $\lfloor \frac 34 x - \frac 14 \rfloor$,$x\geq 0\land x\bmod 4 = 0$ 时需要补上 $1$,$x<0$ 时需要补上 $-1$。
	
	用 \verb[(x >> 31 & 1) | !(x & 3)[ 判断,需要 $9$ 个运算符。
\begin{lstlisting}
int trueThreeFourths(int x) {
	int s = x >> 31 & 1;
	return x + (~x >> 2) + (s | !(x & 3));
}
\end{lstlisting}

	\section{isPower2}
	
	判断 $x$ 是否是 $2$ 的次幂。
	
	由 \verb|lowbit| 的性质可知,$x$ 二进制位 $1$ 只能有一个,满足 \verb|x & (x - 1) = 0|。
	
	发现需要特判 \verb|0| 和 \verb|1 << 31| 两种情况,用符号位判断或 \verb|!x| 判断较劣,可以构造出 \verb|!(x << 1)| 的判断方法,此时需要 $7$ 个运算符。
	
	考虑利用计算的 \verb|x + ~0|,发现两个数分别变为 \verb|0xffffffff| 和 \verb|0x7fffffff|。由于合法的情况最大只能是 \verb|1 << 30|,直接用 \verb|(x + ~0) >> 30| 判断即可,只需要 $6$ 个运算符。
\begin{lstlisting}
int isPower2(int x) {
	int y = x + ~0;
	return !((y >> 30) + (x & y));
}
\end{lstlisting}
	
	\section{float\_i2f}
	
	将 $x$ 转换为 \verb|float|,求出转换后 \verb|unsigned| 下的二进制位,可以使用所有运算符、\verb|unsigned| 及其范围内常数和 \verb|if|,\verb|while|。
	
	初始 \verb|int| 的阶码为 $127+30$,若 $x\geq 0$,将 $x$ 转为 \verb|unsigned|,令 \verb|exp = 0x4e800000|;否则,将 $-x$ 转为 \verb|unsigned|,令 \verb|exp = 0xce800000|。
	
	令其为 $ux$,找到 $ux$ 最高位的 $1$,移位过程中每一位有 \verb|ux <<= 1, exp -= 0x800000|,并用 \verb|if(ux & 0x17f) uf += 0x80;| 来判断尾数四舍六入的情况,此时 \verb|exp + (ux >> 8)| 即为答案。
	
	若进位导致 \verb|ux >> 8| 为 $0$,需要在 \verb|exp| 上再补 \verb|0x01000000| 的阶码。
	
	可以把判断符号位写在循环内,同时需要特判 $0$(不要 \verb|if(!x)|),需要 $14$ 个运算符。
\begin{lstlisting}
unsigned float_i2f(int x) {
	unsigned ux = x;
	unsigned e = 0x4e800000;
	int op = 0;
	if(x) {
		while(1) {
			if(ux & 0x80000000) {
				if(op)
				break;
				else {
					ux = -x;
					e = 0xce800000;
				}
			}
			else {
				ux <<= 1;
				e -= 0x800000;
			}
			op = 1;
		}
		if(ux & 0x17f)
		ux += 0x80;
		ux >>= 8;
		return e + (ux ? ux : 0x01000000);
	}
	return 0;
}
\end{lstlisting}
	
	\section{howManyBits}
	
	计算出表示 $x$ 的最少补码位数。
	
	对于正数,最少补码位数即最高位 $1$ 的位数 $+1$;对于负数,为最高位 $0$ 的位数 $+1$。可以用 \verb|x ^= x << 1| 来将负数的 $0$ 转为 $1$,此时最高位 $1$ 的位数即为答案。
	
	对于最高位的 $1$,二分 $5$ 次查找,最后 $1$ 次可以直接判断,重点在于优化单次查找的运算符数。
	
	直接 \verb|!!(x >> 16) << 4| 由于两个 \verb|!|,含累加答案单次需要 $6$ 个运算符。一种优化是将初始答案改为 $s=31$,采用异或的方式累加,可以节省掉一个 \verb|!|,单次需要 $5$ 个运算符。
	
	共需要 $26$ 个运算符。
\begin{lstlisting}
int howManyBits(int x) {
	int s = 31, y, p;
	x ^= x << 1;
	y = x >> 16; p = !y << 4;
	s ^= p; x <<= p;
	y = x >> 24; p = !y << 3;
	s ^= p; x <<= p;
	y = x >> 28; p = !y << 2;
	s ^= p; x <<= p;
	y = x >> 30; p = !y << 1;
	s ^= p; x <<= p;
	s ^= !(x >> 31);
	return s + 1;
}
\end{lstlisting}
	
	还有优化空间。不对 $x$ 进行移位操作,直接对 $ans$ 进行异或计算,最后累加上构造移位的答案,单次只需要 $4$ 个运算符,总共只需要 $22$ 个运算符。
	
\begin{lstlisting}
int howManyBits(int x) {
	int ans;
	x = x ^ (x << 1);
	ans = !(x >> 16) << 4;
	ans ^= 24;
	ans ^= !(x >> ans) << 3;
	ans ^= 4;
	ans ^= !(x >> ans) << 2;
	ans += ~0x5b >> (x >> ans & 30) & 3;
	return ans + 1;
}
\end{lstlisting}
	
	\section{float\_half}
	
	给出浮点数 $f$ 的 \verb|unsigned| 表达,计算 $0.5\times f$ 的值,如果 $f$ 为 \verb|nan| 返回本身。可以使用所有运算符、\verb|unsigned| 及其范围内常数和 \verb|if|,\verb|while|。
	
	对于 \verb|nan| 和 \verb|inf|,直接返回本身。对于非规格化数,尾数右移 $1$ 位并处理舍入,可同时处理 $0$,用 \verb|if(exp <= 0x800000)| 判断。对于规格化数,阶码减 $1$ 即可。
	
	舍入需要满足 \verb|op = (uf & 3) == 3| 的情况,结果为 \verb!s | ((uf ^ s) + op) >> 1!。
	
	利用 \verb|uf << 1 >> 1| 去除符号位,可以优化为 \verb! s | ((uf << 1) + (uf & 3)) >> 2!,只有 \verb|uf & 3 == 3| 时才会实质上 $+1$,省去一个 \verb|==|,需要 $12$ 个运算符。
\begin{lstlisting}
unsigned float_half(unsigned uf) {
	unsigned s = uf & 0x80000000;
	unsigned e = uf & 0x7f800000;
	int op = uf & 3;
	if(e == 0x7f800000)
		return uf;
	else if(e <= 0x800000)
		return s | ((uf << 1) + op) >> 2;
	else
		return uf - 0x800000;
}
\end{lstlisting}
	
\end{document}